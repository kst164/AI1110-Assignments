\documentclass[journal, 12pt, twocolumn]{IEEEtran}

\usepackage{amsmath}
\usepackage{graphicx}
\usepackage{tfrupee}

\title{Assignment 2 - ICSE 12 2018 - Q4}
\author{Kartheek Tammana}

\begin{document}

\newcommand{\mydet}[1]{\ensuremath{\begin{vmatrix}#1\end{vmatrix}}}

\maketitle

\textbf{Question:}

Use properties of determinant to solve for $x$:

\begin{center}
    $\mydet{ x + a & b & c \\ c & x + b & a \\ a & b & x + c } = 0$ and $x \neq 0$
\end{center}

\textbf{Solution:}
\begin{equation}
    \mydet{
        x + a & b & c \\
        c & x + b & a \\
        a & b & x + c
    } = 0
\end{equation}
    Transforming $C_1 \rightarrow C_1 + C_2 + C_3$,
\begin{align}
    \mydet{
        x + a + b + c & b & c \\
        x + a + b + c & x + b & a \\
        x + a + b + c & b & x + c
    } &= 0 \\
    \implies (x + a + b + c)
    \mydet{
        1 & b & c \\
        1 & x + b & a \\
        1 & b & x + c
    } &= 0
\end{align}
    Transforming $R_3 \rightarrow R_3 - R_1$,
\begin{equation}
    (x + a + b + c)
    \mydet{
        1 & b & c \\
        1 & x + b & a \\
        0 & 0 & x
    } = 0
\end{equation}
    Transforming $R_2 \rightarrow R_2 - R_1$,
\begin{equation}
    (x + a + b + c)
    \mydet{
        1 & b & c \\
        0 & x & a - c \\
        0 & 0 & x
    } = 0
\end{equation}
Expanding the determinant, we get
\begin{equation}
    (x + a + b + c)x^2 = 0
\end{equation}
Since $x \neq 0$, we have
\begin{equation}
    x + a + b + c = 0
\end{equation}
And so we have our answer,
\begin{equation}
    x = -(a + b + c)
\end{equation}

\end{document}
