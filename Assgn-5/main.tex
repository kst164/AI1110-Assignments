\documentclass{beamer}

\usetheme{CambridgeUS}

\usepackage{amsmath}
\usepackage{graphicx}
\usepackage{tfrupee}

% FOR TABLE (according to table generated by ssconvert)
\def\inputGnumericTable{}
\usepackage[latin1]{inputenc}
\usepackage{color}
\usepackage{array}
\usepackage{longtable}
\usepackage{calc}
\usepackage{multirow}
\usepackage{hhline}
\usepackage{ifthen}
\usepackage{lscape}

\providecommand{\pr}[1]{\ensuremath{\Pr\left(#1\right)}}

\title{Assignment 5 \\ CBSE Class 11 Probability \\ Ex 16.3 Q7}
\author{Kartheek Tammana}
\date{\today}
\logo{\large \LaTeX{}}

\begin{document}

\begin{frame}
    \titlepage
\end{frame}

\logo{}

\begin{frame}{Outline}
    \tableofcontents
\end{frame}

\section{Question}
\begin{frame}{Question}
A fair coin is tossed four times, and a person wins \rupee 1 for each head and loses \rupee 1.50 for
each tails that turns up.

From the sample space calculate how many different amounts of money you can have after four tosses
and the probability of having each of the amounts.
\end{frame}

\section{Theory}
\begin{frame}{Theory}
    Let the binomial random variable $X \in \{0, 1, 2, 3, 4\}$ denote the number of heads, and let
    $Y$ denote the net gain in rupees.
\end{frame}

\section{Calculations}
\begin{frame}{Calculations}
    Since rolling heads gives \rupee $1$, and tails gives \rupee $-1.5$, we have that
    \begin{equation}
        Y = (1)(X) + (-1.5)(4 - X) = 2.5 X - 6
    \end{equation}

    Since $X$ is a binomial variable, we know that,
    \begin{equation}
        \pr{X=i} = \binom{4}{i} p^{i} (1-p)^{4-i}
    \end{equation}

    where $p$ is the probability of getting heads. Since the coin is fair, we have
    $p = \frac{1}{2}$, and so,
    \begin{equation}
        \pr{X=i} = \binom{4}{i} {\left(\frac{1}{2}\right)}^{4}
    \end{equation}

\end{frame}

\section{Values}
\begin{frame}{Values}
    Substituting the values for $X$, we get the following probabilities.

    % ssconvert table is not making columns wide enough, and converts decimal numbers into integers
    % ss convert not working, so using tabular instead

    \centering
    \begin{table}
    \begin{tabular}{|l|l|l|}
        \hline
        $X$ & $Y = 2.5 X - 6$ & $\pr{X = i}$ \\
        \hline
        0 & -6 & 0.0625 \\
        1 & -3.5 & 0.25 \\
        2 & -1 & 0.375 \\
        3 & 1.5 & 0.25 \\
        4 & 4 & 0.0625 \\
        \hline
    \end{tabular}
    \caption{Probabilities of different gains}
    \end{table}

    %\begin{table}[!htb]
        %\input{tables/probs.tex}
        %\caption{}
        %\label{table:data}
    %\end{table}
\end{frame}

\section{Graph}
\begin{frame}{Graph}
    \begin{figure}[!ht]
        \includegraphics[width=\textheight]{figs/plot.png}
        \caption{Probability Mass Function}
        \label{figure:1}
    \end{figure}
\end{frame}

\end{document}
