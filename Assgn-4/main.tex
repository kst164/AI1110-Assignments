\documentclass[journal, 12pt, twocolumn]{IEEEtran}

\usepackage{amsmath}
\usepackage{graphicx}
\usepackage{enumitem}

% FOR TABLE (according to table generated by ssconvert)
\def\inputGnumericTable{}
\usepackage[latin1]{inputenc}
\usepackage{color}
\usepackage{array}
\usepackage{longtable}
\usepackage{calc}
\usepackage{multirow}
\usepackage{hhline}
\usepackage{ifthen}
\usepackage{lscape}

\providecommand{\pr}[1]{\ensuremath{\Pr\left(#1\right)}}

\title{Assignment 4 \\ CBSE Class 10 Probability \\ Example 2}
\author{Kartheek Tammana}

\begin{document}

\maketitle

\textbf{Question:}

A bag contains a red ball, a blue ball and a yellow ball, all the balls being of the same size.
Kritika takes out a ball from the bag without looking into it. What is the probability that she
takes out the

\begin{enumerate}[label=(\roman*)]
    \item yellow ball?
    \item red ball?
    \item blue ball?
\end{enumerate}

\textbf{Solution:}

Let the random variable $X \in \{0,1,2\}$ denote the outcome of the experiment, where $X = 0,1,2$
denote the event of choosing the yellow, red, and blue balls respectively.

Since all the balls are identical, each event is equally likely. So we have

\begin{equation}
    \pr{X=0} = \pr{X=1} = \pr{X=2}
\end{equation}

And since the 3 events are exhaustive,

\begin{equation}
    \pr{X=0} + \pr{X=1} + \pr{X=2} = 1
\end{equation}

And so we have the theoretical solution:

\begin{equation}
    \pr{X=0} = \pr{X=1} = \pr{X=2} = \frac{1}{3}
\end{equation}

The program \texttt{./codes/sim.py} simulates this problem experimentally.
\end{document}
