\documentclass[journal, 12pt, twocolumn]{IEEEtran}

\usepackage{amsmath}
\usepackage{graphicx}
\usepackage{enumitem}
\usepackage{listings}
\usepackage[justification=centering]{caption}

\lstset{
%language=C,
basicstyle=\ttfamily,
frame=single,
breaklines=true,
columns=fullflexible
}

\providecommand{\pr}[1]{\ensuremath{\Pr\left(#1\right)}}

\title{Random Numbers}
\author{Kartheek Tammana}
\date{\today}

\begin{document}

\maketitle

\section{Uniform Random Variables}
Let $U$ be a uniform random variable between 0 and 1.
\begin{enumerate}[label=\arabic{section}.\arabic*]
    \item
        Generate $10^6$ samples of $U$ using a C program and save into a file called uni.dat
        \\
        \textbf{Solution} Download the following C file.
        \begin{lstlisting}
wget https://raw.githubusercontent.com/kst164/AI1110-Assignments/main/rand_nums/codes/gen_uniform.c
        \end{lstlisting}
        And run it using
        \begin{lstlisting}
cc gen_uniform.c
./a.out > uni.dat
        \end{lstlisting}

    \item
        Load the uni.dat file into python and plot the empirical CDF of $U$ using the samples in uni.dat.
        \\
        \textbf{Solution} Download the following files.
        \begin{lstlisting}
wget https://raw.githubusercontent.com/kst164/AI1110-Assignments/main/rand_nums/codes/cdf_uni.py
wget https://raw.githubusercontent.com/kst164/AI1110-Assignments/main/rand_nums/codes/uni.dat
        \end{lstlisting}
        And run it using
        \begin{lstlisting}
python3 cdf_uni.py
        \end{lstlisting}
        It will generate the plot in Figure \eqref{fig:cdf_uni}.
        \begin{figure}[!ht]
            \includegraphics[width=\columnwidth]{figs/cdf_uni.png}
            \caption{CDF of $U$}
            \label{fig:cdf_uni}
        \end{figure}

    \item
        Find a theoretical expression for the CDF of $U$.
        \\
        \textbf{Solution}
        The PDF of $U$ is given by
        \begin{equation}
            p_U(x) = 
            \begin{cases}
                1 & 0 \leq x \leq 1 \\
                0 & \text{otherwise}
            \end{cases} \label{eq:pu}
        \end{equation}
        And so we can find the CDF,
        \begin{align}
            F_U(x) &= \int_{-\infty}^{x}{p_U(x) dx} \\
            F_U(x) &=
            \begin{cases}
                0 & x < 0 \\
                x & 0 \leq x \leq 1 \\
                1 & x > 1
            \end{cases} \label{eq:fu}
        \end{align}

    \item
        Write a C program to find the mean and variance of $U$.
        \\
        \textbf{Solution} Download the following files.
        \begin{lstlisting}
wget https://raw.githubusercontent.com/kst164/AI1110-Assignments/main/rand_nums/codes/mean_var.c
wget https://raw.githubusercontent.com/kst164/AI1110-Assignments/main/rand_nums/codes/uni.dat
        \end{lstlisting}
        And run it using
        \begin{lstlisting}
cc mean_var.c
./a.out < uni.dat
        \end{lstlisting}
        It will give the output
        \begin{lstlisting}
Mean    : 0.499631
Variance: 0.083320
        \end{lstlisting}

    \item
        Verify your result theoretically given that
        \begin{equation}
            E\left[U^k\right] = \int_{-\infty}^{\infty}{x^k d F_U(x)}
        \end{equation}
        \\
        \textbf{Solution}
        We use the fact that
        \begin{equation}
            d F_U(x) = p_U(x) dx
        \end{equation}

        So from eq \eqref{eq:pu}, we have
        \begin{align}
            E\left[U^k\right] &= \int_{-\infty}^{+\infty}{x^k p_U(x) dx} \\
            &= \int_{0}^{1}{x^k dx} \\
            &= \left(\frac{x^{k+1}}{k+1}\right)\Big|_0^1 \\
            &= \frac{1}{k+1} \label{eq:ek}
        \end{align}

        Now using eq \eqref{eq:ek}, we can find the mean of $U$
        \begin{equation}
            \mu = E\left[U\right] = \frac{1}{2}
        \end{equation}
        and the variance
        \begin{align}
            \text{var}\left[U\right] &= E\left[U^2\right] - E\left[U\right]^2 \\
            &= \frac{1}{3} - \left(\frac{1}{2}\right)^2 \\
            &= \frac{1}{12} = 0.83333..
        \end{align}
        We see that the theoretical mean and variance match with the experimental values.

\end{enumerate}

\section{Central Limit Theorem}

Let $X$ be a random variable defined as

\begin{equation}
    X = \sum_{i=1}^{12}{U_i - 6}
\end{equation}

\begin{enumerate}[label=\arabic{section}.\arabic*]
    \item
        Generate $10^6$ samples of X using a C program, and save into a file called gau.dat
        \\
        \textbf{Solution} Download the following C file.
        \begin{lstlisting}
wget https://raw.githubusercontent.com/kst164/AI1110-Assignments/main/rand_nums/codes/gen_gaussian.c
        \end{lstlisting}
        And run it using
        \begin{lstlisting}
cc gen_gaussian.c
./a.out > gau.dat
        \end{lstlisting}

    \item
        Load the gau.dat file into python and plot the empirical CDF of $X$ using the samples in gau.dat.
        What properties does the CDF have?
        \\
        \textbf{Solution} Download the following files.
        \begin{lstlisting}
wget https://raw.githubusercontent.com/kst164/AI1110-Assignments/main/rand_nums/codes/cdf_gau.py
wget https://raw.githubusercontent.com/kst164/AI1110-Assignments/main/rand_nums/codes/gau.dat
        \end{lstlisting}
        And run it using
        \begin{lstlisting}
python3 cdf_gau.py
        \end{lstlisting}
        It will generate the plot in Figure \eqref{fig:cdf_gau}.
        \begin{figure}[!ht]
            \includegraphics[width=\columnwidth]{figs/cdf_gau.png}
            \caption{CDF of $X$}
            \label{fig:cdf_gau}
        \end{figure}
        \\
        Properties of the CDF:
        \begin{itemize}
            \item $F_X(x) = \frac{1}{2 \pi} \int_{-\infty}^{x}{\exp\left(-\frac{u^2}{2}\right)du}$
            \item $\lim\limits_{x \to -\infty} F_X(x) = 0$
            \item $\lim\limits_{x \to \infty} F_X(x) = 1$
            \item $F_X(0) = \frac{1}{2}$
            \item $F_X(x) + F_X(-x) = 1$
        \end{itemize}

    \item
        Load the gau.dat file into python and plot the empirical PDF of $X$ using the samples in gau.dat.
        What properties does the PDF have?
        \\
        \textbf{Solution} Download the following files.
        \begin{lstlisting}
wget https://raw.githubusercontent.com/kst164/AI1110-Assignments/main/rand_nums/codes/pdf_gau.py
wget https://raw.githubusercontent.com/kst164/AI1110-Assignments/main/rand_nums/codes/gau.dat
        \end{lstlisting}
        And run it using
        \begin{lstlisting}
python3 pdf_gau.py
        \end{lstlisting}
        It will generate the plot in Figure \eqref{fig:pdf_gau}.
        \begin{figure}[!ht]
            \includegraphics[width=\columnwidth]{figs/pdf_gau.png}
            \caption{PDF of $X$}
            \label{fig:pdf_gau}
        \end{figure}
        \\
        Properties of the PDF:
        \begin{itemize}
            \item $F_X(x) = F_X(-x)$ i.e., symmetric about 0
            \item PDF is bell shaped
            \item Peak of PDF is also the mean
        \end{itemize}

    \item
        Write a C program to find the mean and variance of $X$.
        \\
        \textbf{Solution} Download the following files.
        \begin{lstlisting}
wget https://raw.githubusercontent.com/kst164/AI1110-Assignments/main/rand_nums/codes/mean_var.c
wget https://raw.githubusercontent.com/kst164/AI1110-Assignments/main/rand_nums/codes/gau.dat
        \end{lstlisting}
        And run it using
        \begin{lstlisting}
cc mean_var.c
./a.out < gau.dat
        \end{lstlisting}
        It will give the output
        \begin{lstlisting}
Mean    : 0.000635
Variance: 0.999490
        \end{lstlisting}

    \item
        Given that
        \begin{equation}
            p_X(x) = \frac{1}{\sqrt{2 \pi}} \exp \left(-\frac{x^2}{2}\right)
        \end{equation}
        repeat the above exercise theoretically.
        \\
        \textbf{Solution}
        The mean is given by
        \begin{align}
            E\left[X\right] &= \int_{-\infty}^{\infty}{x p_X(x) dx} \\
            &= \int_{-\infty}^{\infty}{\frac{x}{\sqrt{2 \pi}} \exp\left(-\frac{x^2}{2}\right) dx}
        \end{align}
        Since this is the integral of an odd function over an odd interval, and the function goes to zero as $x$ diverges,
        \begin{equation}
            E\left[U\right] = 0
        \end{equation}

        To calculate variance of $X$
        \begin{align}
            \text{var}(X) &= E\left[X - E[X]\right]^2 \\
            &= E\left[X^2\right] \\
            &= \int_{-\infty}^{\infty}{x^2 p_X(x) dx} \\
            &= \int_{-\infty}^{\infty}{\frac{x^2}{\sqrt{2 \pi}} \exp\left(-\frac{x^2}{2}\right) dx} \\
            &= \frac{1}{\sqrt{2 \pi}} \int_{-\infty}^{\infty}{x \cdot x \exp\left(-\frac{x^2}{2}\right) dx}
        \end{align}

        Integrating by parts, we get
        \begin{align}
            \text{var}(X) &= \frac{1}{\sqrt{2 \pi}}\left(-x \text{exp}\left(\frac{-x^2}{2}\right) + \int{\text{exp}\left(\frac{-x^2}{2}\right) dx}\right) \Bigg|_{-\infty}^{\infty} \\
            &= \frac{1}{\sqrt{2 \pi}} \int_{-\infty}^{\infty}{\text{exp}\left(\frac{-x^2}{2}\right) dx} \\
        \end{align}
        Substituting the Gaussian integral,
        \begin{equation}
            \text{var}(X) = 1
        \end{equation}

\end{enumerate}

\section{From Uniform to Other}

\begin{enumerate}[label=\arabic{section}.\arabic*]
    \item
        Generate samples of
        \begin{equation}
            V = -2 \ln(1 - U)
        \end{equation}
        and plot its CDF
        \\
        \textbf{Solution} Download the following files.
        \begin{lstlisting}
wget https://raw.githubusercontent.com/kst164/AI1110-Assignments/main/rand_nums/codes/cdf_v.py
wget https://raw.githubusercontent.com/kst164/AI1110-Assignments/main/rand_nums/codes/uni.dat
        \end{lstlisting}
        And run it using
        \begin{lstlisting}
python3 cdf_v.py
        \end{lstlisting}
        It will generate the plot in Figure \eqref{fig:cdf_v}.
        \begin{figure}[!ht]
            \includegraphics[width=\columnwidth]{figs/cdf_v.png}
            \caption{}
            \label{fig:cdf_v}
        \end{figure}

    \item
        Find a theoretical expression for $F_V(x)$
        \\
        \textbf{Solution}
        \begin{align}
            F_V(x) &= \pr{x \leq V} \\
            &= \pr{x \leq -2 \log (1 - U)} \\
            &= \pr{\log (1 - U) \leq \frac{-x}{2}} \\
            &= \pr{1 - U \leq \exp \left(\frac{-x}{2}\right)} \\
            &= \pr{1 - \exp \left(\frac{-x}{2}\right) \leq U} \\
            &= F_U \left(1 - \exp \left(\frac{-x}{2}\right)\right)
        \end{align}

        We know $F_U(x)$ from eq \eqref{eq:fu}, so we have
        \begin{align}
            F_V(x) &= F_U \left(1 - \exp \left(\frac{-x}{2}\right)\right) \\
            &= \begin{cases}
                0 & 1 - \exp \left(\frac{-x}{2}\right) < 0 \\
                1 - \exp \left(\frac{-x}{2}\right) & 0 < 1 - \exp \left(\frac{-x}{2}\right) < 1 \\
                1 & 1 - \exp \left(\frac{-x}{2}\right) > 1
            \end{cases}
        \end{align}
        Simplifying, we get
        \begin{equation}
            F_V(x) = \begin{cases}
                0 & x < 0 \\
                1 - \exp \left(\frac{-x}{2}\right) & x \geq 0
            \end{cases}
        \end{equation}
\end{enumerate}

\end{document}
