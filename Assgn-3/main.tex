\documentclass[journal, 12pt, twocolumn]{IEEEtran}

\usepackage{amsmath}
\usepackage{graphicx}

% FOR TABLE (according to table generated by ssconvert)
\def\inputGnumericTable{}
\usepackage[latin1]{inputenc}
\usepackage{color}
\usepackage{array}
\usepackage{longtable}
\usepackage{calc}
\usepackage{multirow}
\usepackage{hhline}
\usepackage{ifthen}
\usepackage{lscape}


\title{Assignment 3 \\ CBSE Class 9 Statistics \\ Exercise 14.3 Q6}
\author{Kartheek Tammana}

\begin{document}

\maketitle

\textbf{Question:}

The following table gives the distribution of students of two sections according to the marks
obtained by them.

\begin{table}[!htb]
    \input{tables/data.tex}
    \caption{}
    \label{table:data}
\end{table}

Represent the marks of the students of both the sections on the same graph by two frequency
polygons. From the two polygons compare the performance of the two sections.

\textbf{Solution:}

The frequency polygon is as seen in Fig. \ref{fig:graph}. It was generated by \texttt{./codes/plot.py}.

\begin{figure}[!ht]
    \centering
    \includegraphics[width=\columnwidth]{./figs/freq_poly.png}
    \caption{Frequency polygons showing marks of students in each section}
    \label{fig:graph}
\end{figure}

From the graph, we see that on average, Section A has more higher scoring students, and Section B
has more lower scoring students. So we can say that the students of Section A did better on average
than the students of Section B.
\end{document}
